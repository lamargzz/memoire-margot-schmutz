\chapter*{Introduction}

% Ajouter manuellement cette section à la table des matières
\addcontentsline{toc}{chapter}{Introduction}

L’assurance vie occupe une place centrale dans le paysage financier français, avec un encours atteignant 1 989 milliards d’euros à fin 2024. Cette croissance soutenue, portée par une collecte nette de +29,4 milliards d’euros, s’inscrit dans un contexte économique incertain, où les épargnants recherchent à la fois sécurité et rendement. Pour les assureurs, cette dynamique implique des exigences accrues en matière de pilotage financier, en particulier dans le cadre du régime prudentiel Solvabilité II (S2), qui impose une évaluation rigoureuse et régulière de la solidité financière de l’entreprise. Le respect du ratio de solvabilité (le Solvency Capital Requirement (SCR) rapporté aux fonds propres éligibles) devient alors un enjeu central, tant en termes réglementaires que stratégiques.

Dans ce contexte, les modèles actuariels et financiers traditionnels, qui reposent majoritairement sur des techniques stochastiques, jouent un rôle clé pour estimer les indicateurs S2. Toutefois, leur complexité computationnelle et leur dépendance à un grand nombre d’hypothèses peuvent constituer un frein à une exploitation rapide, flexible et anticipative des données. Par ailleurs, la volatilité accrue des marchés et la complexité croissante des portefeuilles exigent des outils de prédiction plus agiles et capables de mieux capturer les relations non linéaires entre les multiples facteurs de risque. Ces constats ouvrent la voie à l’introduction de techniques de machine learning (ML), en tant que solutions alternatives ou complémentaires pour anticiper l’évolution du ratio S2, notamment dans le cadre du pilier 2 de Solvabilité II, où le stress testing et les projections internes jouent un rôle déterminant.

Ce mémoire se propose donc d’étudier la capacité des modèles de machine learning à prédire le ratio S2, dans le but d’offrir une lecture plus rapide et potentiellement plus précise de la solvabilité d’un assureur à court et moyen terme. Cette problématique s’inscrit dans une logique d’optimisation des processus de pilotage et de décision stratégique, en explorant dans quelle mesure des approches d’apprentissage supervisé peuvent anticiper les variations du ratio de solvabilité à partir de données internes et économiques, sans recourir à des simulations stochastiques coûteuses en temps et en ressources.

L’approche adoptée dans ce mémoire sera principalement quantitative et expérimentale. Elle s’articulera autour de la construction d’un jeu de données synthétique (ou réel, à déterminer), incluant les principales composantes explicatives à l'actif et au passif du portefeuille. Plusieurs algorithmes de machine learning seront ensuite testés (forêts aléatoires, gradient boosting, réseaux de neurones, etc.) afin d’évaluer leur capacité prédictive du ratio S2. L’objectif ne sera pas seulement d’obtenir un bon score de prédiction, mais aussi de comprendre les facteurs explicatifs les plus influents, en s’appuyant sur des techniques d’interprétabilité.

Ce mémoire s’organisera en cinq parties successives. La première partie posera le cadre théorique et réglementaire du régime Solvabilité II, en particulier les fondements du SCR, et des exigences de capital. La deuxième partie introduira les bases du machine learning supervisé, en précisant les conditions d’application dans un contexte financier et assurantiel. La troisième partie présentera la démarche méthodologique retenue pour constituer les données d’apprentissage et de test, ainsi que les indicateurs d’évaluation de la performance des modèles. La quatrième partie détaillera les résultats des différentes modélisations, en comparant les performances selon les algorithmes utilisés. Enfin, la cinquième partie proposera une lecture critique de ces résultats, notamment en termes de robustesse, d’interprétabilité, et de pertinence opérationnelle, tout en identifiant les limites et les perspectives d’amélioration du recours au machine learning dans un cadre réglementaire tel que Solvabilité II.