\chapter{Apports du machine learning pour la rationalisation des écrats}

\section{Problématique de régression dans le cadre du BE}
\subsection{Formulation du problème}
\subsection{Pourquoi le machine learning est adapté à ce contexte}
\subsection{Positionnement par rapport aux approche traditionnelles}

\section{Présentation des modèles de machine learning utilisés}
\subsection{Modèles de référence}
\subsection{Modèles non linéaires}
\subsection{Avantages et limites de chaque modèles dans le cadre d'étude}

\section{Méthodologie d'apprentissage}
\subsection{Séparation des données}
\subsection{Choix des hyperparamètres}
\subsection{Prévention de surapprentissage}

\section{Critère de performance}
\subsection{Indices retenus}
\subsection{Comparaison des modèles}
\subsection{Lecture des performances}
