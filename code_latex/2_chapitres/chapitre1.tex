\chapter{Cadre théorique et réglementaire du régime Solvabilité II}
\label{chap:contexte}
\newpage
\section{Solvabilité II : principes et objectifs}
\label{sec:spec_av}

\subsection{Historique de la directive européenne}

\subsection{Les trois piliers de S2 : exigences de capital, gouvernance, et transparence}


\section{Définition et calcul du ratio de solvabilité S2}
\label{sec:s2}


\subsection{Une mesure synthétique de la solidité financière d'un assureur}

Définition du ratio S2 : Fonds propres éligibles / SCR

Interprétation économique du ratio (seuil réglementaire de 100 %, zone de confort, zone d’alerte)

Rôle central du ratio dans la communication avec les régulateurs et les investisseurs

\subsection{Le Solvency Capital Requirement (SCR) : estimation des riques extrêmes}

Logique de calcul : capital requis pour absorber un choc extrême avec un niveau de confiance de 99,5 % sur un an

Méthodes de calcul : formule standard vs modèle interne partiel ou complet

Décomposition du SCR : risque de marché, risque de souscription, risque de crédit, risque opérationnel

\subsection{Les fonds propres éligibles : capacité à couvrir les pertes}

Classification des fonds propres : Tier 1, Tier 2, Tier 3

Critères d’éligibilité et ajustements réglementaires

Impact des réévaluations d’actifs, des plus-values latentes et des instruments hybrides

\subsection{Un indicateur sensible aux chocs économiques et aux arbitrages de gestion}

Sensibilité du ratio aux taux d’intérêt, à la volatilité, aux spreads de crédit

Mécanismes de gestion du ratio : couverture du SCR, allocation d’actifs, revalorisation du passif

Enjeux stratégiques : pilotage dynamique de la solvabilité, gestion de la marge de manœuvre

\section{Enjeux de pilotage de la solvabilité}
\label{sec:gse}


\subsection{Rôle de l'anticipation du ratio dans la gestion stratégique}


\subsubsection{Limites opérationnelles des modèles actuels}
