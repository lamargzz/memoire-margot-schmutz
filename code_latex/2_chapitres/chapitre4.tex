\chapter{Analyse des résultats et rationnalisation des écarts}


\section{Qualité de la podélisation du BE stochastique}
\subsection{Capacité des modèles à reproduire le BE stochastique}
\subsection{Comparaison avec le BE déterministe seul}
\subsection{Apports du machine learning}

\section{Analyse des variables explicatives majeurs}
\subsection{Variables dominantes dans l'explication du BE stochastique}
\subsection{Rôle du BE déterministe}
\subsection{Impact des caractéristiques du passif}

\section{Lecture des écarts BE déterministe/BE stochastique}
\subsection{Identification des principaux facteurs d'écarts}
\subsection{Analyse par classification}
\subsection{Mise en évidence des effets non linéaires}
 
\section{Apports pour le traçabilité et la compréhension des résultats}
\subsection{Amélioration de la lisibilité des écarts}
\subsection{Comtribution à la documentation des modèles}
\subsection{Intérêts pour la gouvernance}