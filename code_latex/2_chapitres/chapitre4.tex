\chapter{Super chapitre 4 de Margot !}

% Ce chapitre a pour objectif de définir et de mettre en œuvre un protocole d'analyse rigoureux visant à sélectionner la méthode d'agrégation la plus adaptée à notre étude. Le choix s'appuiera sur une évaluation comparative de plusieurs approches candidates, en considérant des critères clés tels que la fidélité des indicateurs prudentiels (BE et SCR), la performance en termes de temps de calcul et la complexité de mise en œuvre.

% La démarche consistera d'abord à présenter les différentes méthodes d'agrégation envisagées. Ensuite, nous appliquerons chacune de ces méthodes à un portefeuille de référence afin de quantifier les écarts de résultats par rapport à une modélisation sans agrégation.

% À l'issue de cette analyse comparative, la méthode jugée la plus performante sera retenue. Elle sera ensuite utilisée pour conduire les analyses de sensibilité détaillées dans le chapitre suivant.

% \section{Présentation des Méthodes d'Agrégation candidates}
%     \subsection{Approches par clustering (K-means, DBSCAN/HDBSCAN)}
%     % Votre texte ici...
%     \subsection{Autres approches (MP par âge, MP Mémoire Amine Ben Fadhel, etc.)}
%     % Votre texte ici...

% \section{Définition du Protocole de Test Comparatif}
%     \subsection{Constitution des portefeuilles de test}
%     % Votre texte ici...
%     \subsection{Définition des critères de sélection : fidélité des indicateurs (BE/SCR), performance et temps de calcul}
%     % Votre texte ici...

% \section{Analyse Comparative et Choix de la Méthode Optimale}
%     \subsection{Synthèse des performances pour chaque méthode candidate}
%     % Votre texte ici...
%     \subsection{Justification du choix de la méthode retenue pour l'analyse de sensibilité}
%     % Votre texte ici...
% % \chapter{Protocole d'Analyse : Agrégation et Scénarios de Sensibilité}

% % % Introduction du chapitre : Expliquer que ce chapitre pose toute la méthodologie
% % % qui sera appliquée dans la partie suivante. C'est le "comment" de l'analyse.

% % \section{Méthodologie et Sélection du Modèle d'Agrégation}
% % % Objectif : Décrire le processus complet qui mène au choix d'UNE méthode d'agrégation.

% % \subsection{Description des Méthodes d'Agrégation Candidates}
% % % Reprendre votre description technique des méthodes (K-means, DBSCAN, etc.)

% % \subsection{Tests de Performance et Analyse comparative}
% % % Fusionner la présentation des portefeuilles de test et l'analyse
% % \subsubsection{Présentation des portefeuilles de test}
% % % Décrire les portefeuilles utilisés spécifiquement pour comparer les méthodes d'agrégation.

% % \subsubsection{Critères d'évaluation et résultats des tests}
% % % Analyser les résultats (BE, SCR, temps de calcul) pour chaque méthode.
% % % Inclure ici la réflexion sur l'optimisation du nombre de Model Points.

% % \subsection{Choix du Modèle d'Agrégation Optimal pour les métriques S2}
% % % Conclure cette section en justifiant le choix d'UNE méthode spécifique
% % % au regard des résultats précédents et de sa compatibilité avec les architectures modernes.

% % \section{Définition des Scénarios de Sensibilité}
% % % Objectif : Décrire précisément les tests qui seront menés sur les portefeuilles.

% % \subsection{Création des Portefeuilles de Test via le Générateur}
% % % Expliquer comment le générateur est utilisé pour créer les portefeuilles de base
% % % ainsi que leurs variations pour les tests de sensibilité.

% % \subsection{Description des Chocs et Modifications Appliqués}
% % % Détailler les chocs (positifs/négatifs) sur les variables clés.
% % % Décrire le scénario d'ajout d'un nouveau produit (caractéristiques, volume, etc.).
