\chapter{Modèle ALM interne et construction de la base de données}

\section{Le modèle ALM interne}

Au sein de Forsides, le modèle d’ALM s'appelle SALLTO (Solvency Asset Liability Life TOols). Il s’agit d’un outil développé et codé en interne en C\#. Ce modèle permet de représenter l’évolution temporelle d’une compagnie d’assurance vie en intégrant explicitement les interactions entre l’actif et le passif, à partir de scénarios économiques.

Pour fonctionner, le modèle nécessite un fichier d’input dans lequel sont renseignées différentes hypothèses. Ces hypothèses concernent à la fois l’actif et le passif, ainsi que la politique de gestion de la compagnie d’assurance étudiée. Le fichier d’entrée inclut également les scénarios économiques, générés indépendamment par le générateur de scénarios économiques (GSE), qui servent de base aux projections réalisées par le modèle.

\begin{figure}[H]
    \centering
    \includegraphics[width=0.9\textwidth]{code_latex/images/Schema_SALLTO.png}
    \caption{Schéma de fonctionnement de SALLTO}
\end{figure}

\subsection{Générateur de scénarios économiques}

Théorie mathématique derrière

\subsection{Fonctionnement de SALLTO}

Etapes de fonctionnement

\section{Description du portfeuille étudié}
\subsection{}


\section{Méthodologie de la construction de la base de données}

\subsection{Définition de la variable cible et des variables explicatives retenues}

L’objectif principal de ce mémoire est de comprendre et d’expliquer les mécanismes conduisant aux différences de valorisation entre un BE déterministe et un BE stochastique. L’enjeu n’est pas de comparer ces deux approches d’un point de vue normatif, mais d’analyser les facteurs qui contribuent aux écarts observés, afin d’en améliorer la traçabilité, la justification et la compréhension dans un cadre réglementaire. Afin d’aborder la problématique sous des angles complémentaires, deux bases de données distinctes ont été construites. Elles reposent sur un même socle de données explicatives, mais se distinguent par le choix de la variable cible.

La première base de données consiste à considérer le BE stochastique comme variable à expliquer, à partir d’un ensemble de variables explicatives majoritairement liées au passif, incluant notamment des éléments issus du cadre déterministe. Cette approche permet d’évaluer dans quelle mesure les caractéristiques du portefeuille et les hypothèses déterministes contribuent à la formation du niveau du BE stochastique.

La seconde base de données adopte une approche centrée sur l’analyse des écarts, en définissant comme variable cible la différence entre le Best Estimate stochastique et le Best Estimate déterministe. Cette formulation vise à isoler et à analyser plus directement les facteurs expliquant les divergences entre les deux valorisations, en mettant en évidence les effets liés à la prise en compte de l’incertitude, des non-linéarités et des options financières.

L’utilisation conjointe de ces deux approches permet ainsi de proposer une lecture actuarielle approfondie des résultats, en apportant des éléments de compréhension sur le passage du BE déterministe au BE stochastique. Elle s’inscrit dans une démarche de transparence et de gouvernance des modèles, en cohérence avec les exigences de Solvabilité II.

Les variables explicatives retenues sont majoritairement liées aux caractéristiques du passif, conformément à l’objectif du mémoire, qui vise à analyser les écarts de valorisation du point de vue des engagements d’assurance.

\subsection{Construction de la base de données}

Le jeu de données utilisé dans ce mémoire est construit à partir d’une démarche de sensibilités appliquée à SALLTO. Cette approche consiste à faire varier de manière contrôlée certaines hypothèses du modèle, principalement liées au passif, afin d’observer l’impact de ces variations sur la valorisation des engagements d’assurance.

L’objectif de cette démarche est de générer un ensemble d’observations structurées permettant d’analyser, dans un cadre maîtrisé, les relations entre les variables explicatives issues des sensibilités et les résultats de Best Estimate. Chaque observation du jeu de données correspond à une configuration donnée des hypothèses du modèle, définie par un ensemble de valeurs prises par les variables soumises à sensibilité.

Les sensibilités sont réalisées à une date de valorisation donnée, sur un périmètre de passif constant. En dehors des variables explicitement soumises à sensibilité, l’ensemble des hypothèses est maintenu identique sur l’ensemble des calculs. Cette hypothèse garantit que les variations observées sur les résultats de valorisation proviennent exclusivement des paramètres étudiés.

Pour chaque configuration testée, le SALLTO permet de calculer :

un Best Estimate déterministe ;

un Best Estimate stochastique.

Le Best Estimate déterministe constitue à la fois un résultat de valorisation à part entière et une information explicative essentielle pour l’analyse des résultats stochastiques.

Le jeu de données repose sur un ensemble de variables explicatives comprenant à la fois des variables issues des sensibilités et le Best Estimate déterministe.


Les sensibilités réalisées portent principalement sur des variables du passif, complétées par une variable relative à l’actif. Les variables explicatives soumises à sensibilité sont les suivantes :

taux servi ;

rachats conjoncturels ;

rachats structurels ;

âge des assurés ;

taux minimal garanti ;

ancienneté des contrats ;

courbe de taux utilisée pour la valorisation de l’actif.


En complément des variables soumises à sensibilité, le Best Estimate déterministe est intégré au jeu de données en tant que variable explicative. Il constitue une synthèse des caractéristiques du portefeuille et des hypothèses retenues dans le cadre déterministe, et permet de capturer une part importante de l’information expliquant le niveau du Best Estimate stochastique.

L’inclusion du BE déterministe parmi les variables explicatives vise à analyser dans quelle mesure celui-ci permet d’expliquer le BE stochastique, ainsi que les limites de cette explication lorsque des effets non linéaires ou liés à l’incertitude sont présents.

Pour chacune des variables soumises à sensibilité, cinq sensibilités sont définies :

une configuration centrale, correspondant aux hypothèses de référence ;

deux sensibilités à la hausse ;

deux sensibilités à la baisse.

Ces variations sont choisies de manière symétrique autour du scénario central et permettent de couvrir une plage de valeurs suffisamment large pour faire apparaître des effets non linéaires, tout en restant cohérentes.


\subsection{Justification des variables retenues}
