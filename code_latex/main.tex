\documentclass[a4paper]{report} % Changé de 'article' à 'report'

%% Language and font encodings
\usepackage[english, french]{babel}
\usepackage[utf8]{inputenc}
\usepackage[T1]{fontenc}

%% Sets page size and margins
\usepackage{geometry}
\usepackage{lipsum}              % Pour du texte d'exemple (facultatif)

%% Useful packages
\usepackage{amsmath, amssymb}
\usepackage{amssymb} % Redundant if amsmath is used, but kept as in original
\usepackage{graphicx}
\usepackage[colorlinks=true, allcolors=black]{hyperref}
\usepackage{float}
\usepackage{tikz}
\usepackage{calc}
\usepackage{pgfplots}
\usepackage{setspace}
\usepackage{titlesec}
\usepackage{parskip}
\usepackage{tabularx}
\usepackage{booktabs}
\usepackage{enumitem}
\usepackage{xcolor}
\definecolor{forsides}{HTML}{A100FF}
\usetikzlibrary{positioning, shapes.geometric, arrows.meta, shapes.misc, backgrounds, fit, shadows}
\usetikzlibrary{shapes, arrows, positioning}
\setlist[itemize]{itemsep=8pt, topsep=2pt} % Applique ces réglages à toutes les listes itemize
\setcounter{secnumdepth}{3} %pour que le niveau 3 (subsubsection soit numeroté)
\setcounter{tocdepth}{2} %pour que le niveau 2 (subsection apparaisse dans la table des matieres)
% Personnalisation des titres de chapitres
% Met le mot "Chapitre" et son numéro en gras et très grand (\Huge)
% et le titre du chapitre également en \Huge
% Redéfinition complète du format des pages de chapitre


% --- 1. Apparence des titres (police, taille, gras...) ---

\titleformat{\chapter}[display]
  {\normalfont\Huge\bfseries} % Apparence générale : gras et centré
  {\vspace*{50pt}\chaptertitlename\ \thechapter} % Espace au-dessus + Texte "Chapitre X"
  {25pt} % Espace vertical entre "Chapitre X" et le titre du chapitre
  {\Huge} % Taille de la police pour le titre du chapitre

\titleformat{\section}
  {\cleardoublepage\normalfont\LARGE\bfseries} % Police grande et en gras
  {\thesection}
  {1em}
  {}

\titleformat{\subsection}
  {\normalfont\large\bfseries} % Police "large" et en gras
  {\thesubsection}
  {1em}
  {}

\titleformat{\subsubsection}
  {\normalfont\normalsize\bfseries} % Police normale et en gras
  {\thesubsubsection}
  {1em}
  {}


% --- 2. Espacement avant et après les titres ---

% Syntaxe : \titlespacing*{commande}{gauche}{AVANT}{APRES}

\titlespacing*{\section}
  {0pt}      % gauche
  {10ex}      % avant (l'espace que vous vouliez ajouter)
  {3ex}    % après

\titlespacing*{\subsection}
  {0pt}      % gauche
  {6ex}    % avant
  {2.5ex}    % après

\titlespacing*{\subsubsection}
  {0pt}      % gauche
  {4ex}   % avant
  {2.5ex}    % après

\pgfplotsset{compat=1.14}

\setlength{\intextsep}{16pt} % Espace autour des figures dans le texte

%% Added packages for new layout
\usepackage{fancyhdr} % For custom headers and footers

\newcommand\fillin[1]{\makebox[#1]{\dotfill}}


\begin{document}
% La page de garde
\begin{titlepage} % Utiliser l'environnement titlepage pour la page de garde
% Définir une géométrie spécifique pour la page de titre pour éviter le débordement
\newgeometry{inner=1cm, outer=1cm, top=1cm, bottom=0.5cm, includefoot=false, includehead=false}
\pagenumbering{roman} % La numérotation commence ici si désiré pour les pages liminaires
\thispagestyle{empty} % La page de garde ne doit pas avoir d'en-tête/pied de page

% --- START OF FIRST PAGE CONTENT (Logos d'origine rétablis) ---
\begin{tikzpicture}[remember picture, overlay]
% IMPORTANT: Ensure these image files exist in the 'images' subfolder and are valid!
% Logos d'origine de la première page
\node[xshift=-7cm, yshift=-3.1cm] at (current page.north) {\includegraphics[scale=0.11]{images/logo_ida.png}};
\node[below = 0.5cm] at (current page.north) {\includegraphics[scale = 0.11]{images/ucbl.jpg}};
\node[xshift=6.8cm, yshift=-3.3cm] at (current page.north) {\includegraphics[scale=0.11]{images/isfa_logo.jpg}};
\end{tikzpicture}

\vspace*{2.5cm} % Réduction de l'espace vertical

{\Large
{\bfseries
\begin{center}
Mémoire présenté le : \\[0.3cm]
pour l'obtention du Diplôme Universitaire d'actuariat de l'ISFA \\
et l'admission à l'Institut des Actuaires
\end{center}
}
\vspace*{0.5cm} % Réduction de l'espace vertical

\noindent Par : Margot SCHMUTZ \\[0.3cm]
\noindent Titre : Prédictibilité du ratio S2 par machine learning
\\[0.3cm]
\noindent Confidentialité : \quad $\boxtimes$ NON \qquad $\square$ (Durée : $\square$ 1 an \quad  $\square$ 2 ans)
\\
\textit{Les signataires s'engagent à respecter la confidentialité indiquée ci-dessus}
\vspace*{0.5cm}

\noindent
\begin{minipage}{0.6\textwidth}
\begin{tabular}{p{7cm}p{4cm}}
\textit{Membres présents du jury de l'Institut des Actuaires} & \textit{Signature} \\[1.2cm]
\fillin{6cm} & \\[0.8cm]
\fillin{6cm} & \\[0.8cm]
\fillin{6cm} & \\[0.8cm]
\textit{Membres présents du jury de l'ISFA} & \\[1.2cm]
\fillin{6cm} & \\[0.8cm]
\fillin{6cm} & \\[0.8cm]
\fillin{6cm} & \\[0.8cm]
\end{tabular}
\rule{0mm}{4.6cm} % Ensure minipage height
\end{minipage}%
\begin{minipage}{0.4\textwidth}
\textit{Entreprise : FORSIDES} \\
\textit{Nom :} \\[0.3cm]
\textit{Signature :} \\[0.5cm]
\textit{Directeur de mémoire en entreprise :} \\
\textit{Nom : Daniel ZERBIB}  \\[0.3cm]
\textit{Signature :} \\[0.5cm]
\textit{Invité :} \\
\textit{Nom :} \\[0.3cm]
\textit{Signature :} \\[0.5cm]
\textit{\textbf{Autorisation de publication et de mise en ligne sur un site de diffusion de documents actuariels} (après expiration de l'éventuel délai de confidentialité)}\\[0.2cm]
Signature du responsable entreprise \\
\framebox[7cm]{\rule{0mm}{1.5cm}}
Signature du candidat \\
\framebox[7cm]{\rule{0mm}{1.5cm}}
\end{minipage}
}
% --- END OF FIRST PAGE CONTENT ---
\end{titlepage}
\restoregeometry % Restaurer la géométrie précédente avant de la redéfinir pour le reste du document.

% Redéfinir la géométrie pour le reste du document après la page de garde
\newgeometry{left=2.54cm,right=2.54cm,top=3.54cm,bottom=3.54cm}
\onehalfspacing
% Augmenter headheight pour accommoder des logos plus grands
\setlength{\headheight}{50pt} % Ajusté pour des logos de 1.6cm de hauteur

% Configurer fancyhdr pour les en-têtes et pieds de page personnalisés
\pagestyle{fancy}
\fancyhf{} % Effacer tous les champs d'en-tête et de pied de page

% En-têtes des pages suivantes avec les logos demandés (taille doublée)
% IMPORTANT: Assurez-vous que images/forsides_logo.png existe !
\fancyhead[L]{\includegraphics[height=1.6cm]{images/forsides_logo.png}} % Logo forsides à gauche
\fancyhead[R]{\includegraphics[height=1.6cm]{images/isfa_logo.jpg}}   % Logo ISFA à droite

\fancyfoot[C]{\sffamily \thepage}

\renewcommand{\headrulewidth}{0pt}
\renewcommand{\footrulewidth}{0pt}

% Redéfinir le style 'plain' pour qu'il soit identique à 'fancy' afin que les logos apparaissent sur les premières pages de chapitre (taille doublée)
\fancypagestyle{plain}{%
  \fancyhf{}%
  \fancyhead[L]{\includegraphics[height=1.6cm]{images/forsides_logo.png}}% Logo forsides à gauche
  \fancyhead[R]{\includegraphics[height=1.6cm]{images/isfa_logo.jpg}}%   Logo ISFA à droite
  \fancyfoot[C]{\sffamily \thepage}%
  \renewcommand{\headrulewidth}{0pt}%
  \renewcommand{\footrulewidth}{0pt}%
}


% Contenu principal en une seule colonne
\tableofcontents

% IMPORTANT: Ensure all included .tex files exist in the specified paths
% and do not contain LaTeX errors themselves!
\chapter*{Résumé}
% Ajouter manuellement cette section à la table des matières
\addcontentsline{toc}{chapter}{Résumé}
Petit résumé en français de mon mémoire ?
\chapter*{Abstract}
% Ajouter manuellement cette section à la table des matières
\addcontentsline{toc}{chapter}{Abstract}

It's a brief sum up in english !
\chapter*{Remerciements}
% Ajouter manuellement cette section à la table des matières
\addcontentsline{toc}{chapter}{Remerciements}

\chapter*{Synthèse}
% Ajouter manuellement cette section à la table des matières
\addcontentsline{toc}{chapter}{Synthèse}

Un long résumé en français de mon mémoire 
\chapter*{Synthesis}
% Ajouter manuellement cette section à la table des matières
\addcontentsline{toc}{chapter}{Synthesis}

Un long résumé en anglais de mon mémoire


\clearpage
\pagenumbering{arabic}

\chapter*{Introduction}

% Ajouter manuellement cette section à la table des matières
\addcontentsline{toc}{chapter}{Introduction}

L’assurance vie occupe une place centrale dans le paysage financier français, avec un encours atteignant 1 989 milliards d’euros à fin 2024. Cette croissance soutenue, portée par une collecte nette de +29,4 milliards d’euros, s’inscrit dans un contexte économique incertain, où les épargnants recherchent à la fois sécurité et rendement. Pour les assureurs, cette dynamique implique des exigences accrues en matière de pilotage financier, en particulier dans le cadre du régime prudentiel Solvabilité II (S2), qui impose une évaluation rigoureuse et régulière de la solidité financière de l’entreprise. Le respect du ratio de solvabilité (le Solvency Capital Requirement (SCR) rapporté aux fonds propres éligibles) devient alors un enjeu central, tant en termes réglementaires que stratégiques.

Dans ce contexte, les modèles actuariels et financiers traditionnels, qui reposent majoritairement sur des techniques stochastiques, jouent un rôle clé pour estimer les indicateurs S2. Toutefois, leur complexité computationnelle et leur dépendance à un grand nombre d’hypothèses peuvent constituer un frein à une exploitation rapide, flexible et anticipative des données. Par ailleurs, la volatilité accrue des marchés et la complexité croissante des portefeuilles exigent des outils de prédiction plus agiles et capables de mieux capturer les relations non linéaires entre les multiples facteurs de risque. Ces constats ouvrent la voie à l’introduction de techniques de machine learning (ML), en tant que solutions alternatives ou complémentaires pour anticiper l’évolution du ratio S2, notamment dans le cadre du pilier 2 de Solvabilité II, où le stress testing et les projections internes jouent un rôle déterminant.

Ce mémoire se propose donc d’étudier la capacité des modèles de machine learning à prédire le ratio S2, dans le but d’offrir une lecture plus rapide et potentiellement plus précise de la solvabilité d’un assureur à court et moyen terme. Cette problématique s’inscrit dans une logique d’optimisation des processus de pilotage et de décision stratégique, en explorant dans quelle mesure des approches d’apprentissage supervisé peuvent anticiper les variations du ratio de solvabilité à partir de données internes et économiques, sans recourir à des simulations stochastiques coûteuses en temps et en ressources.

L’approche adoptée dans ce mémoire sera principalement quantitative et expérimentale. Elle s’articulera autour de la construction d’un jeu de données synthétique (ou réel, à déterminer), incluant les principales composantes explicatives à l'actif et au passif du portefeuille. Plusieurs algorithmes de machine learning seront ensuite testés (forêts aléatoires, gradient boosting, réseaux de neurones, etc.) afin d’évaluer leur capacité prédictive du ratio S2. L’objectif ne sera pas seulement d’obtenir un bon score de prédiction, mais aussi de comprendre les facteurs explicatifs les plus influents, en s’appuyant sur des techniques d’interprétabilité.

Ce mémoire s’organisera en cinq parties successives. La première partie posera le cadre théorique et réglementaire du régime Solvabilité II, en particulier les fondements du SCR, et des exigences de capital. La deuxième partie introduira les bases du machine learning supervisé, en précisant les conditions d’application dans un contexte financier et assurantiel. La troisième partie présentera la démarche méthodologique retenue pour constituer les données d’apprentissage et de test, ainsi que les indicateurs d’évaluation de la performance des modèles. La quatrième partie détaillera les résultats des différentes modélisations, en comparant les performances selon les algorithmes utilisés. Enfin, la cinquième partie proposera une lecture critique de ces résultats, notamment en termes de robustesse, d’interprétabilité, et de pertinence opérationnelle, tout en identifiant les limites et les perspectives d’amélioration du recours au machine learning dans un cadre réglementaire tel que Solvabilité II.

\titleformat{\chapter}[display]
  {\normalfont\Huge\bfseries} % Apparence générale : gras et centré
  {\vspace*{50pt}\chaptertitlename\ \thechapter} % Espace au-dessus + Texte "Chapitre X"
  {25pt} % Espace vertical entre "Chapitre X" et le titre du chapitre
  {\Huge} % Taille de la police pour le titre du chapitre
  [\vfill\newpage] % Reste de la page vide et saut de page

\chapter{Cadre théorique et réglementaire du régime Solvabilité II}
\label{chap:contexte}
\newpage
\section{Solvabilité II : principes et objectifs}
\label{sec:spec_av}

\subsection{Historique de la directive européenne}

\subsection{Les trois piliers de S2 : exigences de capital, gouvernance, et transparence}


\section{Définition et calcul du ratio de solvabilité S2}
\label{sec:s2}


\subsection{Une mesure synthétique de la solidité financière d'un assureur}

Définition du ratio S2 : Fonds propres éligibles / SCR

Interprétation économique du ratio (seuil réglementaire de 100 %, zone de confort, zone d’alerte)

Rôle central du ratio dans la communication avec les régulateurs et les investisseurs

\subsection{Le Solvency Capital Requirement (SCR) : estimation des riques extrêmes}

Logique de calcul : capital requis pour absorber un choc extrême avec un niveau de confiance de 99,5 % sur un an

Méthodes de calcul : formule standard vs modèle interne partiel ou complet

Décomposition du SCR : risque de marché, risque de souscription, risque de crédit, risque opérationnel

\subsection{Les fonds propres éligibles : capacité à couvrir les pertes}

Classification des fonds propres : Tier 1, Tier 2, Tier 3

Critères d’éligibilité et ajustements réglementaires

Impact des réévaluations d’actifs, des plus-values latentes et des instruments hybrides

\subsection{Un indicateur sensible aux chocs économiques et aux arbitrages de gestion}

Sensibilité du ratio aux taux d’intérêt, à la volatilité, aux spreads de crédit

Mécanismes de gestion du ratio : couverture du SCR, allocation d’actifs, revalorisation du passif

Enjeux stratégiques : pilotage dynamique de la solvabilité, gestion de la marge de manœuvre

\section{Enjeux de pilotage de la solvabilité}
\label{sec:gse}


\subsection{Rôle de l'anticipation du ratio dans la gestion stratégique}


\subsubsection{Limites opérationnelles des modèles actuels}

\chapter{Modèle ALM interne et construction de la base de données}

\section{Le modèle ALM interne}

Au sein de Forsides, le modèle d’ALM s'appelle SALLTO (Solvency Asset Liability Life TOols). Il s’agit d’un outil développé et codé en interne en C\#. Ce modèle permet de représenter l’évolution temporelle d’une compagnie d’assurance vie en intégrant explicitement les interactions entre l’actif et le passif, à partir de scénarios économiques.

Pour fonctionner, le modèle nécessite un fichier d’input dans lequel sont renseignées différentes hypothèses. Ces hypothèses concernent à la fois l’actif et le passif, ainsi que la politique de gestion de la compagnie d’assurance étudiée. Le fichier d’entrée inclut également les scénarios économiques, générés indépendamment par le générateur de scénarios économiques (GSE), qui servent de base aux projections réalisées par le modèle.

\begin{figure}[H]
    \centering
    \includegraphics[width=0.9\textwidth]{code_latex/images/Schema_SALLTO.png}
    \caption{Schéma de fonctionnement de SALLTO}
\end{figure}

\subsection{Générateur de scénarios économiques}

Théorie mathématique derrière

\subsection{Fonctionnement de SALLTO}

Etapes de fonctionnement

\section{Description du portfeuille étudié}
\subsection{}


\section{Méthodologie de la construction de la base de données}

\subsection{Définition de la variable cible et des variables explicatives retenues}

L’objectif principal de ce mémoire est de comprendre et d’expliquer les mécanismes conduisant aux différences de valorisation entre un BE déterministe et un BE stochastique. L’enjeu n’est pas de comparer ces deux approches d’un point de vue normatif, mais d’analyser les facteurs qui contribuent aux écarts observés, afin d’en améliorer la traçabilité, la justification et la compréhension dans un cadre réglementaire. Afin d’aborder la problématique sous des angles complémentaires, deux bases de données distinctes ont été construites. Elles reposent sur un même socle de données explicatives, mais se distinguent par le choix de la variable cible.

La première base de données consiste à considérer le BE stochastique comme variable à expliquer, à partir d’un ensemble de variables explicatives majoritairement liées au passif, incluant notamment des éléments issus du cadre déterministe. Cette approche permet d’évaluer dans quelle mesure les caractéristiques du portefeuille et les hypothèses déterministes contribuent à la formation du niveau du BE stochastique.

La seconde base de données adopte une approche centrée sur l’analyse des écarts, en définissant comme variable cible la différence entre le Best Estimate stochastique et le Best Estimate déterministe. Cette formulation vise à isoler et à analyser plus directement les facteurs expliquant les divergences entre les deux valorisations, en mettant en évidence les effets liés à la prise en compte de l’incertitude, des non-linéarités et des options financières.

L’utilisation conjointe de ces deux approches permet ainsi de proposer une lecture actuarielle approfondie des résultats, en apportant des éléments de compréhension sur le passage du BE déterministe au BE stochastique. Elle s’inscrit dans une démarche de transparence et de gouvernance des modèles, en cohérence avec les exigences de Solvabilité II.

Les variables explicatives retenues sont majoritairement liées aux caractéristiques du passif, conformément à l’objectif du mémoire, qui vise à analyser les écarts de valorisation du point de vue des engagements d’assurance.

\subsection{Construction de la base de données}

Le jeu de données utilisé dans ce mémoire est construit à partir d’une démarche de sensibilités appliquée à SALLTO. Cette approche consiste à faire varier de manière contrôlée certaines hypothèses du modèle, principalement liées au passif, afin d’observer l’impact de ces variations sur la valorisation des engagements d’assurance.

L’objectif de cette démarche est de générer un ensemble d’observations structurées permettant d’analyser, dans un cadre maîtrisé, les relations entre les variables explicatives issues des sensibilités et les résultats de Best Estimate. Chaque observation du jeu de données correspond à une configuration donnée des hypothèses du modèle, définie par un ensemble de valeurs prises par les variables soumises à sensibilité.

Les sensibilités sont réalisées à une date de valorisation donnée, sur un périmètre de passif constant. En dehors des variables explicitement soumises à sensibilité, l’ensemble des hypothèses est maintenu identique sur l’ensemble des calculs. Cette hypothèse garantit que les variations observées sur les résultats de valorisation proviennent exclusivement des paramètres étudiés.

Pour chaque configuration testée, le SALLTO permet de calculer :

un Best Estimate déterministe ;

un Best Estimate stochastique.

Le Best Estimate déterministe constitue à la fois un résultat de valorisation à part entière et une information explicative essentielle pour l’analyse des résultats stochastiques.

Le jeu de données repose sur un ensemble de variables explicatives comprenant à la fois des variables issues des sensibilités et le Best Estimate déterministe.


Les sensibilités réalisées portent principalement sur des variables du passif, complétées par une variable relative à l’actif. Les variables explicatives soumises à sensibilité sont les suivantes :

taux servi ;

rachats conjoncturels ;

rachats structurels ;

âge des assurés ;

taux minimal garanti ;

ancienneté des contrats ;

courbe de taux utilisée pour la valorisation de l’actif.


En complément des variables soumises à sensibilité, le Best Estimate déterministe est intégré au jeu de données en tant que variable explicative. Il constitue une synthèse des caractéristiques du portefeuille et des hypothèses retenues dans le cadre déterministe, et permet de capturer une part importante de l’information expliquant le niveau du Best Estimate stochastique.

L’inclusion du BE déterministe parmi les variables explicatives vise à analyser dans quelle mesure celui-ci permet d’expliquer le BE stochastique, ainsi que les limites de cette explication lorsque des effets non linéaires ou liés à l’incertitude sont présents.

Pour chacune des variables soumises à sensibilité, cinq sensibilités sont définies :

une configuration centrale, correspondant aux hypothèses de référence ;

deux sensibilités à la hausse ;

deux sensibilités à la baisse.

Ces variations sont choisies de manière symétrique autour du scénario central et permettent de couvrir une plage de valeurs suffisamment large pour faire apparaître des effets non linéaires, tout en restant cohérentes.


\subsection{Justification des variables retenues}

\chapter{Apports du machine learning pour la rationalisation des écrats}

\section{Problématique de régression dans le cadre du BE}
\subsection{Formulation du problème}
\subsection{Pourquoi le machine learning est adapté à ce contexte}
\subsection{Positionnement par rapport aux approche traditionnelles}

\section{Présentation des modèles de machine learning utilisés}
\subsection{Modèles de référence}
\subsection{Modèles non linéaires}
\subsection{Avantages et limites de chaque modèles dans le cadre d'étude}

\section{Méthodologie d'apprentissage}
\subsection{Séparation des données}
\subsection{Choix des hyperparamètres}
\subsection{Prévention de surapprentissage}

\section{Critère de performance}
\subsection{Indices retenus}
\subsection{Comparaison des modèles}
\subsection{Lecture des performances}

\chapter{Analyse des résultats et rationnalisation des écarts}


\section{Qualité de la podélisation du BE stochastique}
\subsection{Capacité des modèles à reproduire le BE stochastique}
\subsection{Comparaison avec le BE déterministe seul}
\subsection{Apports du machine learning}

\section{Analyse des variables explicatives majeurs}
\subsection{Variables dominantes dans l'explication du BE stochastique}
\subsection{Rôle du BE déterministe}
\subsection{Impact des caractéristiques du passif}

\section{Lecture des écarts BE déterministe/BE stochastique}
\subsection{Identification des principaux facteurs d'écarts}
\subsection{Analyse par classification}
\subsection{Mise en évidence des effets non linéaires}
 
\section{Apports pour le traçabilité et la compréhension des résultats}
\subsection{Amélioration de la lisibilité des écarts}
\subsection{Comtribution à la documentation des modèles}
\subsection{Intérêts pour la gouvernance}
\chapter{Limites, approts et perspectives}

\section{Limites de l'approche}
\subsection{Limites liées aux données}
\subsection{Limites des modèles de machine learning}
\subsection{Dépendance au modèle ALM utilisé}

\section{Apports opérationnels et réglementaires}
\subsection{Aide à l'analyse des résultats}
\subsection{Outils de compréhension complémentaire}
\subsection{Contribution à la transparence}

\section{Perspectives d'amélioration}
\subsection{Extension à d'autres portefeuilles}
\subsection{Intégration d'autres variables}
\subsection{Approfondissement des méthodes d'explicabilité}
\chapter{Conclusion}
% \section{Résumé des résultats}
% \subsection{Synthèse des principaux résultats obtenus}
% \subsection{Impact des méthodes d'agrégation sur les portefeuilles de passifs}

% \section{Perspectives d'amélioration}
% \subsection{Axes d'amélioration pour les générateurs de portefeuilles de passifs}
% \subsection{Évolutions possibles des méthodes d'agrégation et de modélisation ALM}  
% \subsection{Autres domaines d'application des générateurs de portefeuilles de passifs}
% \section{Conclusion Générale et Perspectives}

% Ce mémoire, présenté ici dans sa version intermédiaire, dresse le bilan d'une première phase de recherche consacrée à la construction d'un cadre méthodologique et au développement des outils de modélisation. Les résultats quantitatifs finaux ne sont pas encore établis à ce stade, la priorité ayant été donnée à l'élaboration d'un socle technique robuste. Cette conclusion a donc pour double vocation de définir avec précision la feuille de route pour la finalisation des travaux et de présenter un bilan synthétique des compétences clés acquises chez forsides, qui ont été déterminantes pour ce projet.

% \subsection{Perspectives et Finalisation des Travaux de Recherche}

% Le modèle ALM sous Python, désormais opérationnel, constitue la pierre angulaire de l'analyse à venir. Les prochaines étapes seront dédiées à son exploitation intensive afin de répondre à la problématique du mémoire. Les axes de travail prioritaires sont les suivants :

% \begin{itemize}
%     \item \textbf{Analyse Comparative et Critique des Méthodes d'Agrégation :} Une investigation approfondie de plusieurs méthodes d'agrégation de portefeuilles sera menée. Au-delà d'une simple application mécanique, il s'agira de disséquer les fondements théoriques de chaque approche, leurs hypothèses et leur incidence sur la mesure de la diversification des risques. Une attention particulière sera portée à la \textbf{robustesse} des méthodes, c'est-à-dire leur capacité à produire des indicateurs agrégés stables et convergents, y compris sous des conditions de marché dégradées ou lors de variations des hypothèses de modélisation.

%     \item \textbf{Enrichissement du Générateur de Portefeuilles de Passifs :} La crédibilité des résultats repose sur le réalisme des simulations. Le générateur de portefeuilles sera donc affiné pour intégrer des structures de dépendance plus sophistiquées entre les variables clés (âge des assurés, montant des provisions mathématiques, comportement de rachat, etc.). Cette évolution permettra de simuler des portefeuilles plus fidèles à la réalité d'un assureur et de tester la résilience de nos conclusions dans des scénarios adverses.

%     \item \textbf{Conduite d'une Campagne d'Analyses de Sensibilité :} Une fois les méthodologies d'agrégation validées, une série de tests de sensibilité sera déployée. En appliquant les chocs réglementaires prévus par Solvabilité 2 (choc taux, actions, mortalité, rachat), nous vérifierons la stabilité des agrégations et quantifierons leur impact sur le Solvency Capital Requirement (SCR). L'objectif final est d'offrir une vision claire de la manière dont l'agrégation modifie et stabilise le profil de risque d'un portefeuille.
% \end{itemize}

% La feuille de route pour la finalisation du mémoire est clairement établie et s'inscrit dans la continuité logique des travaux déjà réalisés. Les parties non encore écrites ont un plan clair et détaillé qui montre la robustesse de l'approche. La maîtrise des outils et des concepts acquise jusqu'à présent me permet d'aborder cette dernière phase avec une grande confiance. Une planification rigoureuse et une exécution méthodique assureront l'achèvement des analyses et la rédaction finale dans les délais impartis.

% \subsection{Bilan de l'expérience professionnelle chez forsides : un levier pour le mémoire}



% Mon année en alternance chez forsides a été une expérience fondatrice, m'offrant un cadre idéal pour développer les compétences techniques indispensables à la réalisation de ce mémoire.



% \subsubsection{Développement et industrialisation d'un modèle ALM en Python}

% Au cœur de mes missions se trouvait le développement d'un modèle ALM (Asset-Liability Management) entièrement programmé en Python. Sous la supervision d'un Manager, j'ai pu assimiler les meilleures pratiques de l'ingénierie logicielle appliquée à l'actuariat. Cela inclut la gestion de projet via \textbf{Git}, un système de contrôle de version essentiel pour le travail collaboratif et le suivi rigoureux des modifications du code. J'ai également mis en place des batteries de tests unitaires, notamment via des interfaces avec Excel, pour garantir la fiabilité et la cohérence des résultats du modèle à chaque itération.



% Ce socle technique m'a permis de mener à bien des projets complexes, comme la transformation du modèle ALM développé en un modèle de projection de passifs de retraite. Cette adaptation a exigé une refonte profonde de la logique de modélisation du passif, consolidant ainsi ma compréhension des mécanismes actuariels. J'ai par ailleurs conçu et développé un générateur de portefeuilles de passifs, un outil stratégique pour forsides afin de réaliser des tests de performance et de résistance sur le modèle.

% \subsubsection{Développement d'un modèle de passifs sociaux sous Python}

% Dans le cadre de mon alternance chez forsides, j'ai eu l'opportunité de recréer un logiciel calculant des passifs sociaux en Python. Les passifs sociaux sont des engagements financiers liés aux prestations sociales, tels que les retraites ou les indemnités de départ à la retraite que les entreprises doivent provisionner. Ce projet m'a permis de mettre en pratique mes compétences en programmation tout en approfondissant ma compréhension des enjeux actuels liés à la gestion des ressources humaines et à la prévoyance sociale.

% Ce logiciel a été conçu pour être modulable et facilement intégrable dans les systèmes existants des clients d'forsides, facilitant ainsi son adoption. J'ai également veillé à ce qu'il soit accompagné d'une documentation complète, permettant aux utilisateurs de comprendre rapidement son fonctionnement et de l'utiliser efficacement pour des futures missions sur ce sujet.

% \subsubsection{Maîtrise d'un écosystème technique moderne}

% Cette immersion professionnelle m'a permis de maîtriser un environnement de développement avancé. J'ai travaillé quotidiennement sous \textbf{Linux}, un système d'exploitation open-source prisé pour sa stabilité et sa sécurité, qui est le standard dans de nombreux environnements de calcul scientifique et de serveurs. J'ai également acquis une compétence sur \textbf{Docker}, une technologie de conteneurisation. Un conteneur est une sorte de "boîte" logicielle qui embarque une application et toutes ses dépendances, garantissant ainsi qu'elle s'exécute de manière identique et fiable, quel que soit l'ordinateur ou le serveur sur lequel elle est déployée.



% \subsubsection{Expertise en gestion de données et en calcul distribué (Cloud)}

% La manipulation de grands volumes de données est au cœur des problématiques actuarielles modernes. J'ai pu renforcer mes compétences en utilisant des bibliothèques Python de haute performance comme \textbf{Polars} pour le traitement de données et \textbf{Xlwings} pour automatiser les interactions entre Python et Excel. Mon apprentissage le plus significatif fut la prise en main de \textbf{Snowflake}, une plateforme de données hébergée dans le cloud. Snowflake permet non seulement de stocker et de requêter d'énormes jeux de données via le langage SQL, mais aussi d'exploiter sa puissance de calcul massive pour exécuter des scripts Python, comme le modèle ALM, directement sur la plateforme, optimisant ainsi drastiquement les temps de traitement et les besoins en ressources.



% \subsubsection{Collaboration, autonomie et transmission des savoirs}

% Au-delà des aspects purement techniques, j'ai appris à évoluer au sein d'une équipe projet agile, en participant à des réunions de suivi quotidiennes. J'ai également eu la responsabilité de former plusieurs nouveaux collaborateurs, à la fois sur le fonctionnement technique du modèle ALM et sur les bonnes pratiques de développement à adopter. Cette expérience de transmission a été extrêmement formatrice, renforçant mes capacités de communication et de vulgarisation de concepts complexes.

\titleformat{\chapter}[display]
  {\normalfont\Huge\bfseries} % Apparence générale : gras et centré
  {\vspace*{50pt}\chaptertitlename\ \thechapter} % Espace au-dessus + Texte "Chapitre X"
  {25pt} % Espace vertical entre "Chapitre X" et le titre du chapitre
  {\Huge} % Taille de la police pour le titre du chapitre
  
\include{3_epilogue/annexes}
\newpage
\addcontentsline{toc}{chapter}{Bibliographie} 

\begin{thebibliography}{99} % "99" permet d'aligner si vous avez beaucoup de sources

\bibitem{clustering_book}
DORNAIKA Fadi, HAMAD Denis, CONSTANTIN Joseph, TRONG HOANG Vinh. \textit{Advances in Data Clustering}. Springer, 2024.

\bibitem{goffard_guerrault}
GOFFARD Pierre-Olivier, GUERRAULT Xavier. « Is it optimal to group policyholders by age, gender, and seniority for BEL computations based on model points? ». \textit{European Actuariel Journal}, volume 5, 17 Avril 2015, p. 165-180.

\bibitem{memoire_ben_fadhel}
BEN FADHEL Amine. « Accéleration de l'évaluation de la solvabilité prospective d'un assureur épargne ». \textit{Mémoire pour l'Institut des Actuaires}, 2022.

\bibitem{france_assureurs}
FRANCE ASSUREURS. \textit{L'assurance vie en 2023}. (20 septembre 2024). Consulté le 1er Septembre 2025, sur \url{https://www.franceassureurs.fr/nos-chiffres-cles/assurance-vie/etude-statistique-assurance-vie-2023/}

\end{thebibliography}

\end{document}
